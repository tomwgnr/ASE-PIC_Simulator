\documentclass[12pt,a4paper,titlepage,ngerman,pdftex]{report}
\usepackage[utf8]{inputenc}
\usepackage[german]{babel}
\usepackage[T1]{fontenc}
\usepackage{amsmath}
\usepackage{amsfonts}
\usepackage{amssymb}
\usepackage{graphicx}
\usepackage{acronym}
\usepackage{setspace}
\usepackage{geometry}
\usepackage{caption}
\usepackage{subcaption}
\usepackage{hyperref}
\usepackage{listings}
\usepackage{textcomp}
\geometry{a4paper, top=25mm, left=25mm, right=25mm, bottom=25mm,
    headsep=10mm, footskip=12mm}
\usepackage{xcolor}

\definecolor{mygreen}{rgb}{0,0.6,0}
\definecolor{mygray}{rgb}{0.5,0.5,0.5}
\definecolor{mymauve}{rgb}{0.58,0,0.82}
\definecolor{backcolour}{rgb}{0.827, 0.827, 0.827}
\definecolor{dkblue}{rgb}{0,0,.6}
\definecolor{dkyellow}{cmyk}{0,0,.8,.3}

\lstset{
    language        = php,
    basicstyle      = \small\ttfamily,
    keywordstyle    = \color{dkblue},
    stringstyle     = \color{red},
    identifierstyle = \color{mygreen},
    commentstyle    = \color{gray},
    emph            =[1]{php},
    emphstyle       =[1]\color{black},
    emph            =[2]{if,and,or,else},
    emphstyle       =[2]\color{dkyellow}}

\lstset{ %
    backgroundcolor=\color{backcolour},   % choose the background color
    basicstyle=\footnotesize,        % size of fonts used for the code
    breaklines=true,                 % automatic line breaking only at whitespace
    captionpos=b,                    % sets the caption-position to bottom
    commentstyle=\color{mygreen},    % comment style
    escapeinside={\%*}{*)},          % if you want to add LaTeX within your code
    keywordstyle=\color{blue},       % keyword style
    stringstyle=\color{mymauve},     % string literal stylebreakatwhitespace=false,
    keepspaces=true,
    numbers=left,
    numbersep=5pt,
    showspaces=false,
    showstringspaces=true,
    showtabs=false,
    tabsize=2
}

\begin{document}
%%%%%%%%%%%%%%%%%%%%%%%%%%%%%%%%%%%%%%%%%%%%%%%
%%%%%%%%%%%%%%%%%    TITLE    %%%%%%%%%%%%%%%%%
%%%%%%%%%%%%%%%%%%%%%%%%%%%%%%%%%%%%%%%%%%%%%%%
    \begin{titlepage}
        \centering
        %\includegraphics[width=0.15\textwidth]{pic/picsim_logo_full.png}\par\vspace{1cm}
        {\scshape\LARGE Duale Hochschule Baden-Württemberg \par}
        \vspace{1cm}
        {\scshape\Large Advanced Software Engineering 2 \\--\\ Dokumentation\par}
        \vspace{1.5cm}
        {\huge\bfseries PIC-Simulator\par}
        \vspace{2cm}
        {\Large\itshape David Eymann, Tom Wagner\par}
        \vfill
        Dozent\par
        Daniel \textsc{Lindner}

        \vfill

% Bottom of the page
        {\large \today\par}
    \end{titlepage}

%%%%%%%%%%%%%%%%%%%%%%%%%%%%%%%%%%%%%%%%%%%%%%%
%%%%%%%%%%%%%%%%%  LISTINGS  %%%%%%%%%%%%%%%%%
%%%%%%%%%%%%%%%%%%%%%%%%%%%%%%%%%%%%%%%%%%%%%%%
    \pagenumbering{Roman}
    \tableofcontents
    \listoffigures
    \lstlistoflistings

    \onehalfspacing
    \pagenumbering{arabic}

    \chapter{Projektbeschreibung}\label{ch:projektbeschreibung}
    Als Basis für diese Arbeit dient ein PIC-Simulator, PIC steht hierbei für ein Mikrocontroller von Microchip Technology\footnote{\url{https://en.wikipedia.org/wiki/Microchip_Technology}}.
    Der Simulator ist in C\# geschrieben und mit Windows Forms Erhält er seine Grafische Oberfläche.
    \textbf{Das für diese Abgabe relevante Repository befindet sich unter:} \\ \url{https://github.com/tomwgnr/ASE-PIC_Simulator}\\
    
    

    \section{Einrichtung}\label{sec:einrichtung}

    \chapter{Entwicklung}\label{ch:entwicklung}
    \section{Clean Architecture}\label{sec:cleanarchitecture}
 
    \section{Refactoring}\label{sec:refactoring}

    \section{Unit Tests}

    \subsection{Einsatz von Mocks}

    \subsection{ATRIP-Regeln}

    \subsubsection{Automatic}

    \subsubsection{Thorough}\label{subsec:thorough}

    \subsubsection{Repeatable}

    \subsubsection{Independent}

    \subsubsection{Professional}

    \section{Programming Principles}

    \subsection{SOLID}
    
    \subsubsection{Single responsibility principle}

    \subsubsection{Open/Closed principle}

    \subsubsection{Liskov substitution principle}

    \subsubsection{Interface segregation principle}

    \subsubsection{Dependency inversion principle}

    \subsection{GRASP}

    \subsubsection{High Cohesion}

    \subsubsection{Low Coupling}
    
    \subsection{DRY -- Don't Repeat Yourself}

\end{document}
